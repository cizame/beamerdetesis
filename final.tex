\documentclass[10pt,xcolor={usenames,dvipsnames}]{beamer}
%\documentclass{beamer}
%\documentclass{beamer}
%\usepackage{BeamerColor}
%\usetheme{Berkeley}
%\usepackage[svgnames]{xcolor}

%\usepackage{}
%\usepackage[margin=3cm]{geometry}
\usepackage{amssymb,amsmath,latexsym,amsthm}
\usepackage{amssymb,latexsym}
%\usepackage{wasysym}
\usepackage[spanish,mexico,es-nolayout]{babel}
\usepackage[latin1]{inputenc}
\usepackage{graphicx}
\usepackage{tikz}
\usepackage{tkz-berge}
%\usepackage{color}

%\usepackage{xspace}
%\usepackage{tocbibind}

%\xcolor{dvipsnames}
\usepackage{geometry}
%\geometry{bindingoffset=1cm}

%\usepackage{color}
%\usepackage[usenames,dvipsnames,svgnames,table]{xcolor}
%\usepackage[usenames,dvipsnames]{xcolor}
%\usepackage{tocbibind}

%\usepackage{qtree}
%\mode<beamer>{\setbeamertemplate{blocks}[rounded][shadow=true]}



%\usetheme{Warsaw}
%
%\setbeamertemplate{theorems}[numbered]
\usetheme{Warsaw}
\usecolortheme{seagull}
%\useoutertheme{shadow}
%\useoutertheme{smoothtree}
%\useoutertheme{split}



%\newtheorem*{GT}{Teorema de Grivaux, 2003}
%\newtheorem*{CLT}{Teorema de Can Le, 2010}
%\theoremstyle{Ejemplo}
%\newtheorem*{ER}{Ejemplo: Rolewicz, 1969}
%\theoremstyle{plain}
%\newtheorem*{thm}{Theorema}

%\newtheorem*{prop}{Proposition}

\newtheorem*{teo}{Teorema}
\newtheorem*{cor}{Corolario}
\newtheorem*{prop}{Proposici�n}



\newcommand{\Z}{\mathbb Z}
\newcommand{\N}{\mathbb N}
\newcommand{\C}{\mathbb C}
\newcommand{\R}{\mathbb R}
\newcommand{\Q}{\mathbb Q}
%\newcommand{\ltwo}{\ell^2}
\newcommand{\ip}[2]{\left\langle #1 , #2 \right\rangle }
\newcommand{\norm}[1]{\| #1 \|}

\newcounter{questioncount}
\setcounter{questioncount}{0}

\newcounter{in}
\newcounter{ini}


\usetikzlibrary{arrows,shapes}
% \newenvironment{question}
% {\setbeamercolor{block title}{bg=green}
% \stepcounter{questioncount}
% \begin{block}{Question \arabic{questioncount}}}
% {\end{block}}

%\newenvironment{proposition}
%{\begin{block}{Proposition}}
%{\end{block}}


\setlength{\parskip}{1 em}
\setlength{\parindent}{0 in}
\title{Buscando jaulas c�bicas a partir de gr�ficas de Cayley}
\author{Citlalli Zamora Mej�a}
\institute{UAEH-CIMA-LIMA}

\begin{document}
\pgfdeclarelayer{background}
\pgfsetlayers{background,main}


\maketitle


\section{Conseptos b�sicos}


\begin{frame}
{Definiciones}

\begin{enumerate}
\item Grupo
\pause
\item Grupo sim�trio o de permutaciones
\pause
\item Grupo abeliano
\pause
\item Subgrupo
\pause
\item

\end{enumerate}


\end{frame}


\begin{frame}

\begin{alertblock}<+->{Teorema}
Sean $G$ un grupo y $H$ un subconjunto no vac�o de $G$. Entonces las
  siguientes condiciones son equivalentes:
  \begin{enumerate}
  \item $H$ es un subgrupo de $G$
  \item \begin{enumerate}
    \item Para todo $x,y \in H$, $xy\in H$
    \item Para todo $x\in H$, $x^{-1}\in H$
    \end{enumerate}
  \item Para todo $x,y \in H$ se tiene $xy^{-1}\in H$
  \end{enumerate}\label{teonodemos}
\end{alertblock}

\end{frame}


\begin{frame}

\begin{alertblock}<-+>{Teorema}
La intersecci�n de subgrupos de un grupo tambi�n es un subgrupo. 
\end{alertblock}

\begin{enumerate}
\item Subgrupo generado por un conjunto

\item Orden del grupo

\item Orden de un elemento

\item Isomorfismo de grupos

\item Automorfismo

\end{enumerate}

\end{frame}


\begin{frame}
\begin{alertblock}<-+>{Teorema de Lagrange}
Si $H$ es un subgrupo de un grupo finito $G$, entonces:\label{TeoLagrange}
$$|G|=|H|[G:H]$$
\end{alertblock}
\end{frame}


\section{Jaulas c�bicas}


\begin{frame}{Cota de Moore}


la cota
\end{frame}



\begin{frame}{Cuello Par}

\begin{figure}[htb]
    \centering
    \begin{tikzpicture}[scale=0.3]
      \SetUpVertex[MinSize=.2pt] \SetVertexNoLabel
      \grSubtreeOfCage[RA=8.4,RB=2.5]{3}{6}
       \grEmptyPath[x=-6,y=10,RA=12,prefix=a]{2}
       \grEmptyPath[x=-8,y=12,RA=4,prefix=b]{2}
       \grEmptyPath[x=4,y=12,RA=4,prefix=c]{2}
       \grEmptyPath[x=-9,y=14,RA=2,prefix=d]{4}
       \grEmptyPath[x=3,y=14,RA=2,prefix=e]{4}
\Edges(d0,b0,a0,b1,d3)
\Edges(e0,c0,a1,c1,e3)
\Edges(d1,b0)
\Edges(e1,c0)
\Edges(d2,b1)
\Edges(e2,c1)

\draw(-6,6.5) node {.};\draw(-6,7.5) node {.};\draw(-6,8.5) node {.};
\draw(6,6.5) node {.};\draw(6,7.5) node {.};\draw(6,8.5) node {.};
\draw(-3,10) node {.};\draw(-1,10) node {.};
\draw(1,10) node {.};\draw(3,10) node {.};


  \draw(14,0.3) node {Primer nivel}; \draw(14,2.8) node {Segundo
    nivel};\draw(14,5.3) node {Tercer nivel};\draw(14,10.3) node {Nivel
    $n-2$};\draw(14,12.3) node {Nivel $n-1$};\draw(14,14.3) node {�ltimo nivel};
     %  \grSubtreeOfCage[RA=3,RB=1]{3}{5}
    \end{tikzpicture}
%    \caption{�rbol en una gr�fica 3-regular de cuello par}\label{arbolpar}
\end{figure}


\begin{block}{}
\begin{center}
$\text{cota de Moore} = \sum^{m}_{i=1}2^i$
\end{center}
\end{block}

\end{frame}




\begin{frame}{Cuello impar}
\begin{figure}
    \centering
    \begin{tikzpicture}[scale=0.3]
      \SetUpVertex[MinSize=.2pt] \SetVertexNoLabel
      \grSubtreeOfCage[RA=6,RB=2.5]{3}{5}
       \grEmptyPath[x=-6,y=10,RA=12,prefix=a]{2}
       \grEmptyPath[x=-8,y=12,RA=4,prefix=b]{2}
       \grEmptyPath[x=4,y=12,RA=4,prefix=c]{2}
       \grEmptyPath[x=-9,y=14,RA=2,prefix=d]{4}
       \grEmptyPath[x=3,y=14,RA=2,prefix=e]{4}
\Edges(d0,b0,a0,b1,d3)
\Edges(e0,c0,a1,c1,e3)
\Edges(d1,b0)
\Edges(e1,c0)
\Edges(d2,b1)
\Edges(e2,c1)

\draw(-6,6.5) node {.};\draw(-6,7.5) node {.};\draw(-6,8.5) node {.};
\draw(6,6.5) node {.};\draw(6,7.5) node {.};\draw(6,8.5) node {.};
\draw(-3,10) node {.};\draw(-1,10) node {.};
\draw(1,10) node {.};\draw(3,10) node {.};


  \draw(14,0.3) node {Primer nivel}; \draw(14,2.8) node {Segundo
    nivel};\draw(14,5.3) node {Tercer nivel};\draw(14,10.3) node {Nivel
    $n-2$};\draw(14,12.3) node {Nivel $n-1$};\draw(14,14.3) node {�ltimo nivel};
     %  \grSubtreeOfCage[RA=3,RB=1]{3}{5}
    \end{tikzpicture}
    %\caption{�rbol en una gr�fica 3-regular de cuello impar}\label{arbolimpar}
  \end{figure}

\begin{block}{}
\begin{center}
$\text{cota de Moore} = 1 + 3\sum^m_{i=0} 2^i$
\end{center}
\end{block}


\end{frame}




\begin{frame}{$(3,4)$-jaula}
\begin{figure}
  \centering
  \begin{tikzpicture}[scale=.9]
    \SetUpVertex[MinSize=2pt]\SetVertexNoLabel

 \grSubtreeOfCage[RA=3.2,RB=1.6]{3}{4} \foreach \y in {0,1,2,3}{%
      \setcounter{in}{\y} \stepcounter{in} 

\draw (a1;\y)
      node[below right]{$\alph{in}$}; })

 \begin{pgfonlayer}{background}
        \pause
    \Edge[style={bend right},color=Mulberry](a1;2)(a1;0)
\pause
    \Edge[style={bend right},color=Mulberry](a1;3)(a1;0)
\pause
    \Edge[style={bend right},color=blue](a1;2)(a1;1) \Edge[style={bend left},color=blue](a1;1)(a1;3)
\pause 

 \draw (a0;0) node[below left]{$w_{0,0}$};
 \draw (a0;1) node[below right]{$w_{0,1}$};
  
    \end{pgfonlayer}

  \end{tikzpicture}
%  \caption{} \label{arbol(3,4)}
\end{figure}
\only<5>{
\begin{figure}
  \centering
  \begin{tikzpicture}[scale=.7]
    \SetVertexNoLabel \SetUpVertex[MinSize=2pt]
    \grCompleteBipartite[RA=3,RB=3,RS=3]{3}{3}
  
 \draw (a0) node[below left]{$w_{0,0}$};
 \draw (a1) node[below left]{$c$};
  \draw (a2) node[below left]{$d$};
 \draw (b0) node[above left]{$a$};
\draw (b1) node[above left]{$b$};
\draw (b2) node[above left]{$w_{0,1}$};
\end{tikzpicture}
%  \caption{Bipartita completa $K_{3,3}$.} \label{K_3,3}
\end{figure}}
\end{frame}


\begin{frame}{$(3,5)$-jaula}


\begin{figure}
  \centering
  \begin{tikzpicture}
    \SetUpVertex[MinSize=2pt] \SetVertexNoLabel
    \grSubtreeOfCage[RA=3,RB=1.2]{3}{5} \foreach \y in {0,1,...,5}{%
      \setcounter{in}{\y} \stepcounter{in} \draw (a2;\y) node[below right]{$\alph{in}$}; }
    % \AssignVertexLabel{a2;0}{a}
 \begin{pgfonlayer}{background}
\pause
    \Edges[style={bend left=30},color=Mulberry](a2;0,a2;2) 
\pause
\Edges[style={bend
      left=30},color=Mulberry](a2;0,a2;4) 
\pause
\Edges[style={bend left=30},color=blue](a2;1,a2;3)
    \Edges[style={bend left=30},color=blue](a2;1,a2;5)
\pause
    \Edges[style={bend left=30},color=purple](a2;2,a2;5)
\pause
    \Edges[style={bend left}=30,color=purple](a2;3,a2;4)

% \pause
% \draw (a0;0) node[left]{$w_{0,0}$};
% \draw (a1;0) node[left]{$w_{1,0}$};
% \draw (a1;1) node[left]{$w_{1,1}$};
% \draw (a1;2) node[left]{$w_{1,2}$};
\end{pgfonlayer}
  
\end{tikzpicture}
%  \caption{} \label{jaula(3,5)}
\end{figure}

\end{frame}


\begin{frame}{Gr�fica de Petersen}

\begin{figure}
  \centering
\begin{minipage}[c]{.3\linewidth}
  \begin{tikzpicture}[scale=.7]
    \SetUpVertex[MinSize=2pt] \SetVertexNoLabel
    \grSubtreeOfCage[RA=2.6,RB=1.3]{3}{5} \foreach \y in {0,1,...,5}{%
      \setcounter{in}{\y} \stepcounter{in} \draw (a2;\y) node[below right]{$\alph{in}$}; }
    % \AssignVertexLabel{a2;0}{a}
    \Edges[style={bend left=30},color=Mulberry](a2;0,a2;2) \Edges[style={bend
      left=30},color=Mulberry](a2;0,a2;4) \Edges[style={bend left=30},color=blue](a2;1,a2;3)
    \Edges[style={bend left=30},color=blue](a2;1,a2;5)
    \Edges[style={bend left=30},color=purple](a2;2,a2;5)
    \Edges[style={bend left}=30,color=purple](a2;3,a2;4)
\draw (a0;0) node[left]{$w_{0,0}$};
\draw (a1;0) node[left]{$w_{1,0}$};
\draw (a1;1) node[left]{$w_{1,1}$};
\draw (a1;2) node[left]{$w_{1,2}$};
  \end{tikzpicture}
%  \caption{} \label{jaula(3,5)}

 \end{minipage}
  \hspace{3cm}
  \begin{minipage}[c]{.4\linewidth}
    \centering
  \begin{tikzpicture}[rotate=90,scale=1]
    \SetVertexNoLabel \SetUpVertex[MinSize=2pt] \grPetersen[RA=2,RB=1]
    % \draw(-3,0) node {Petersen};
\draw (a0) node[above]{$a$};
\draw (a1) node[above left]{$c$};
\draw (a2) node[below left]{$f$};
\draw (a3) node[below right]{$w_{1,2}$};
\draw (a4) node[above right]{$e$};
\draw (b0) node[above left]{$w_{1,0}$};
\draw (b1) node[above ]{$w_{1,1}$};
\draw (b2) node[left]{$b$};
\draw (b3) node[right]{$w_{0,0}$};
\draw (b4) node[above]{$d$};
  \end{tikzpicture}
%  \caption{Gr�fica de Petersen.} \label{petersen}\index{Gr�fica de
%    Petersen}

 \end{minipage}
\end{figure}


\end{frame}



\begin{frame}{$(3,6)$-jaula}


\begin{figure}
  \centering
  \begin{tikzpicture}[scale=1.2]
    \SetVertexNoLabel \SetUpVertex[MinSize=2pt]
    \grSubtreeOfCage[RA=5,RB=1]{3}{6}

    \foreach \y in {0,1,...,7}{%
      \setcounter{in}{\y} \stepcounter{in} \draw (a2;\y) node[below
      right]{$\alph{in}$}; } 

\begin{pgfonlayer}{background}
\pause
\Edges[style={bend left=40},color=Mulberry](a2;0,a2;4)
    \Edges[style={bend left=40},color=Mulberry](a2;0,a2;6)
\pause
 \Edges[style={bend
      left=40},color=blue](a2;1,a2;5) \Edges[style={bend
      left=40},color=blue](a2;1,a2;7)
\pause
    \Edges[style={bend left=40},color=purple](a2;2,a2;4) \Edges[style={bend
      left=40},color=purple](a2;2,a2;7)
\pause
\Edges[style={bend left=40},color=red](a2;3,a2;5)
  \Edges[style={bend left=40},color=red](a2;3,a2;6)

\end{pgfonlayer}

  \end{tikzpicture}
 % \caption{} \label{jaula(3,6)}
\end{figure}

\end{frame}



\begin{frame}{Gr�fica de Heawood}

\begin{figure}
  \centering
\begin{minipage}[a]{.2\linewidth}
  \begin{tikzpicture}[scale=.6]

    \SetVertexNoLabel \SetUpVertex[MinSize=2pt]
    \grSubtreeOfCage[RA=4.5,RB=1.5]{3}{6}

    \foreach \y in {0,1,...,7}{%
      \setcounter{in}{\y} \stepcounter{in} \draw (a2;\y) node[below
      right]{$\alph{in}$}; } 
\Edges[style={bend left=40},color=Mulberry](a2;0,a2;4)
    \Edges[style={bend left=40},color=Mulberry](a2;0,a2;6) \Edges[style={bend
      left=40},color=blue](a2;1,a2;5) \Edges[style={bend left=40},color=blue](a2;1,a2;7)
    \Edges[style={bend left=40},color=purple](a2;2,a2;4) \Edges[style={bend
      left=40},color=purple](a2;2,a2;7)

\Edges[style={bend left=40},color=red](a2;3,a2;5)
  \Edges[style={bend left=40},color=red](a2;3,a2;6)


\draw (a0;0) node[below left]{$w_{0,0}$};
\draw (a0;1) node[below right]{$w_{0,1}$};
\draw (a1;0) node[left]{$w_{1,0}$};
\draw (a1;1) node[left]{$w_{1,1}$};
\draw (a1;2) node[left]{$w_{1,2}$};
\draw (a1;3) node[left]{$w_{1,3}$};


  \end{tikzpicture}
 \end{minipage}
  \hspace{3cm}
  \begin{minipage}[b]{.5\linewidth}
    \centering
  \begin{tikzpicture}[scale=.65]
   
    \SetVertexNoLabel \SetUpVertex[MinSize=1pt]
    \grHeawood[RA=3]
\draw (a4) node[above left]{$c$};
\draw (a5) node[above left]{$e$};
\draw (a6) node[above left]{$a$};
\draw (a7) node[left]{$w_{1,0}$};
\draw (a8) node[below left]{$b$};
\draw (a9) node[below left]{$h$};
\draw (a10) node[below left]{$w_{1,3}$};
\draw (a11) node[below right]{$g$};
\draw (a12) node[below right]{$d$};
\draw (a13) node[below right]{$f$};
\draw (a0) node[right]{$w_{1,2}$};
\draw (a1) node[right]{$w_{0,1}$};
\draw (a2) node[right]{$w_{0,0}$};
\draw (a3) node[right]{$w_{1,1}$};
  \end{tikzpicture}
 \end{minipage}
\end{figure}

\end{frame}



\begin{frame}{$(3,7)$-jaula}

\begin{teo}
La jaula $3$-regular de cuello 7 no alcanza su cota de Moore.
\end{teo}

\begin{block}{}
\begin{equation*}
 \text{ cota de Moore para la $(3,7)$-jaula}=22
\end{equation*}
\end{block}
\end{frame}

\begin{frame}{$(3,7)$-jaula}
\begin{figure}
  \centering
  \begin{tikzpicture}[scale=0.5]
    \SetVertexNoLabel 
\SetUpVertex[MinSize=.1pt]
      \grSubtreeOfCage[RA=3,RB=1]{3}{7}  

%% \foreach \y in
 %    {0,1,...,11}{%
 %      \setcounter{in}{\y} \stepcounter{in} \draw (a3;\y) node[below
 %      right]{$\alph{in}$}; }

    \grEmptyPath*[x=-1.5,y=5,RA=2.5,prefix=v]{,}
 %   \AssignVertexLabel {v}{$v_1$,$v_2$}
    % \draw(-2.5,2) node {primer rama}; \draw(0,1.3) node {segunda
    % rama};

\begin{pgfonlayer}{background}
\only<2> {  
\draw(0,-.5) node {$x$};
\Edges(v0,v1)
}
\end{pgfonlayer}

  \end{tikzpicture}
%  \caption{} \label{arbol(3,7)mas2}
\end{figure}

\pause

\begin{teo}
Sea $G$ una gr�fica $3$-regular de cuello $7$ con 24
v�rtices. Entonces existe un v�rtice $x$ para el cual los dos v�rtices
fuera del �rbol que tiene como base a $x$, son adyacentes.
\end{teo}

\end{frame}


\begin{frame}{$(3,7)$-jaula}
\begin{figure}
  \centering
  \begin{tikzpicture}[scale=1]
    \SetVertexNoLabel { \SetUpVertex[MinSize=1.5pt]
      \grSubtreeOfCage[RA=4,RB=1]{3}{7} } \foreach \y in
    {0,1,...,11}{%
      \setcounter{in}{\y} \stepcounter{in} \draw (a3;\y) node[below
      right]{$\alph{in}$}; }
    \grEmptyPath*[x=-2,y=5.4,RA=4,prefix=v]{$v_1$,$v_2$}
    \AssignVertexLabel {v}{$v_1$,$v_2$} \Edges(v0,v1)
  
\begin{pgfonlayer}{background}
\pause
  \Edges[color=Brown](a3;0,v0,a3;4)
\pause
    \Edges[color=Brown](a3;2,v1)
\pause
    \Edges[style={bend left=30},color=red](a3;4,a3;8)
\pause   
 \Edges[style={bend left=30},color=red](a3;3,a3;8)
\pause
    \Edges[style={bend left=30},color=blue](a3;3,a3;6)
\pause
    \Edges[style={bend left=30},color=blue](a3;6,a3;10)
\pause    
    \Edges[style={bend left=30},color=Mulberry](a3;2,a3;11)
    \Edges[style={bend left=30},color=Mulberry](a3;5,a3;11)
    \Edges[style={bend left=30},color=Mulberry](a3;1,a3;5)
    \Edges[style={bend left=30},color=BlueViolet](a3;1,a3;9)
    \Edges[style={bend left=30},color=BlueViolet](a3;7,a3;9)
    \Edges[color=BlueViolet](a3;7,v1) 
    \Edges[style={bend left=30},color=WildStrawberry](a3;0,a3;10)

\end{pgfonlayer}
%    \draw(-3.2,.3) node {Primer rama}; \draw(0,.6) node {Segunda
 %     rama}; \draw(3.2,.3) node {Tercer rama}; \draw(0,-.5) node {x};
  \end{tikzpicture}
\only<4>{
  \caption{Jaula encontrada por McGee.} \label{jaula(3,7)}}
\end{figure}

\end{frame}



\begin{frame}{Jaula encontrada por McGee}

\begin{figure}
  \centering
\begin{minipage}[a]{.19\linewidth}
  \begin{tikzpicture}[scale=.5]


    \SetVertexNoLabel { \SetUpVertex[MinSize=1.5pt]
      \grSubtreeOfCage[RA=4,RB=1.5]{3}{7} } \foreach \y in
    {0,1,...,11}{%
      \setcounter{in}{\y} \stepcounter{in} \draw (a3;\y) node[below
      right]{$\alph{in}$}; }
    \grEmptyPath*[x=-2,y=7,RA=4,prefix=v]{$v_1$,$v_2$}
    \AssignVertexLabel {v}{$v_1$,$v_2$} \Edges(v0,v1)
    \Edges[color=Brown](a3;0,v0,a3;4)
    \Edges[color=Brown](a3;2,v1)
    \Edges[style={bend left=30},color=blue](a3;4,a3;8)
    \Edges[style={bend left=30},color=blue](a3;3,a3;8)
    \Edges[style={bend left=30},color=blue](a3;3,a3;6)
    \Edges[style={bend left=30},color=blue](a3;6,a3;10)
    \Edges[style={bend left=30},color=Mulberry](a3;2,a3;11)
    \Edges[style={bend left=30},color=Mulberry](a3;5,a3;11)
    \Edges[style={bend left=30},color=Mulberry](a3;1,a3;5)
    \Edges[style={bend left=30},color=BlueViolet](a3;1,a3;9)
    \Edges[style={bend left=30},color=BlueViolet](a3;7,a3;9)
    \Edges[color=BlueViolet](a3;7,v1) 
    \Edges[style={bend left=30},color=WildStrawberry](a3;0,a3;10)


\draw (a0;0) node[below left]{$w_{0,0}$};
\draw (a1;0) node[left]{$w_{1,0}$};
\draw (a1;1) node[left]{$w_{1,1}$};
\draw (a1;2) node[left]{$w_{1,2}$};
\draw (a2;0) node[left]{$w_{2,0}$};
\draw (a2;1) node[left]{$w_{2,1}$};
\draw (a2;2) node[left]{$w_{2,2}$};
\draw (a2;3) node[left]{$w_{2,3}$};
\draw (a2;4) node[left]{$w_{2,4}$};
\draw (a2;5) node[left]{$w_{2,5}$};

  \end{tikzpicture}
 \end{minipage}
  \hspace{3cm}
  \begin{minipage}[b]{.5\linewidth}
    \centering
  \begin{tikzpicture}[scale=.44,rotate=90]
    \SetVertexNoLabel \SetUpVertex[MinSize=2pt] \grMcGee[RA=5]%,RB=1]
    

\draw (a0) node[above]{$v_{1}$};
\draw (a1) node[above left]{$a$};
\draw (a2) node[above left]{$k$};
\draw (a3) node[above left]{$g$};
\draw (a4) node[above left]{$w_{2,3}$};
\draw (a5) node[above left]{$w_{1,1}$};
\draw (a6) node[left]{$w_{0,0}$};
\draw (a7) node[below left]{$w_{1,0}$};
\draw (a8) node[below left]{$w_{2,0}$};
\draw (a9) node[below left]{$b$};
\draw (a10) node[below left]{$j$};
\draw (a11) node[below left]{$h$};
\draw (a12) node[below left]{$v_2$};
\draw (a13) node[below right]{$c$};
\draw (a14) node[below right]{$w_{2,1}$};
\draw (a15) node[below right]{$d$};
\draw (a16) node[below right]{$i$};
\draw (a17) node[below right]{$w_{2,4}$};
\draw (a18) node[right]{$w_{1,2}$};
\draw (a19) node[above right]{$w_{2,5}$};
\draw (a20) node[above right]{$l$};
\draw (a21) node[above right]{$f$};
\draw (a22) node[above right]{$w_{2,2}$};
\draw (a23) node[above right]{$e$};

  \end{tikzpicture}
 \end{minipage}
\end{figure}

\end{frame}

\begin{frame}{$(3,8)$-jaula}
\begin{figure}
    \centering
  \begin{tikzpicture}[scale=1.1]
    \SetVertexNoLabel \SetUpVertex[MinSize=2pt]
    \grSubtreeOfCage[RA=5,RB=1]{3}{8}
     \foreach \y in {0,1,...,15}{%
      \setcounter{in}{\y} \stepcounter{in} \draw (a3;\y) node [below
       right]{$\alph{in}$}; }
\pause
    \Edges[style={bend left=50},color=blue](a3;0,a3;8)
    \Edges[style={bend left=50},color=blue](a3;0,a3;12) 
   \Edges[style={bend left=50,color=Mulberry}](a3;1,a3;10)
    \Edges[style={bend left=50},color=Mulberry](a3;1,a3;14)
\pause
    \Edges[style={bend left=50},color=RedViolet](a3;2,a3;9)
    \Edges[style={bend left=50},color=red](a3;3,a3;11)
    \Edges[style={bend left},color=RoyalBlue](a3;4,a3;8)
    \Edges[style={bend left},color=RoyalBlue](a3;6,a3;12)
\pause
\only<4,6>{
    \Edges[style={bend left,line width=3pt}](a3;7,a3;10)
}

\pause
{\only<5>{

    \GraphInit[vstyle=Shade]{
    \Edges[style={bend right=35},color=OliveGreen](a3;7,a3;15)
     \Edges(a3;15,a2;7,a3;14))
\Edges[style={bend left=50,color=Mulberry}](a3;1,a3;10)
    \Edges[style={bend left=50},color=Mulberry](a3;1,a3;14)
    \Edges[style={bend left=40},color=OliveGreen](a3;7,a3;10)
}
}}

\pause
\pause
\only<7>{ 
   \Edges[style={bend left=40,line width=3pt},color=OliveGreen](a3;5,a3;10)
}
 \pause
 \Edges[style={bend left=40},color=OliveGreen](a3;5,a3;10)
   \Edges[style={bend left=40},color=OliveGreen](a3;7,a3;14)
    \Edges[style={bend left=30},color=Mulberry](a3;4,a3;15)
    \Edges[style={bend left=70},color=Mulberry](a3;6,a3;11)
    \Edges[style={bend left=50},color=OrangeRed](a3;3,a3;15)
    \Edges[style={bend left=60},color=OrangeRed](a3;2,a3;13)
    \Edges[style={bend left=60},color=OrangeRed](a3;5,a3;13)
    \Edges[style={bend left=60},color=OrangeRed](a3;7,a3;9)
  \end{tikzpicture}             
\end{figure}

\end{frame}


\begin{frame}{$(3,8)$-jaula de Tutte}
\begin{figure}
  \centering
\begin{minipage}[a]{.19\linewidth}
  \begin{tikzpicture}[scale=.55]

    \SetVertexNoLabel \SetUpVertex[MinSize=2pt]
    \grSubtreeOfCage[RA=5.5,RB=1.2]{3}{8}

    \foreach \y in {0,1,...,15}{%
      \setcounter{in}{\y} \stepcounter{in} \draw (a3;\y) node [below
      right]{$\alph{in}$}; }

    \Edges[style={bend left=50},color=blue](a3;0,a3;8)
    \Edges[style={bend left=50},color=blue](a3;0,a3;12)
    \Edges[style={bend left=50,color=Mulberry}](a3;1,a3;10)
    \Edges[style={bend left=50},color=Mulberry](a3;1,a3;14)
    \Edges[style={bend left=50},color=RedViolet](a3;2,a3;9)
    \Edges[style={bend left=50},color=red](a3;3,a3;11)
    \Edges[style={bend left},color=RoyalBlue](a3;4,a3;8)
    \Edges[style={bend left},color=RoyalBlue](a3;6,a3;12)
    \Edges[style={bend left=40},color=OliveGreen](a3;5,a3;10)
    \Edges[style={bend left=40},color=OliveGreen](a3;7,a3;14)
    \Edges[style={bend left=30},color=Mulberry](a3;4,a3;15)
    \Edges[style={bend left=70},color=Mulberry](a3;6,a3;11)
    \Edges[style={bend left=50},color=OrangeRed](a3;3,a3;15)
    \Edges[style={bend left=60},color=OrangeRed](a3;2,a3;13)
    \Edges[style={bend left=60},color=OrangeRed](a3;5,a3;13)
    \Edges[style={bend left=60},color=OrangeRed](a3;7,a3;9)

\draw (a0;0) node[left]{$w_{0,0}$};
\draw (a0;1) node[right]{$w_{0,1}$};
\draw (a1;0) node[left]{$w_{1,0}$};
\draw (a1;1) node[left]{$w_{1,1}$};
\draw (a1;2) node[left]{$w_{1,2}$};
\draw (a1;3) node[left]{$w_{1,3}$};
\draw (a2;0) node[below]{$w_{2,0}$};
\draw (a2;1) node[below]{$w_{2,1}$};
\draw (a2;2) node[below]{$w_{2,2}$};
\draw (a2;3) node[below]{$w_{2,3}$};
\draw (a2;4) node[below]{$w_{2,4}$};
\draw (a2;5) node[below]{$w_{2,5}$};
\draw (a2;6) node[below]{$w_{2,6}$};
\draw (a2;7) node[below]{$w_{2,7}$};


  \end{tikzpicture}
 \end{minipage}
  \hspace{3cm}
  \begin{minipage}[b]{.5\linewidth}
    \centering
  \begin{tikzpicture}[scale=.43,rotate=95]
    \SetVertexNoLabel \SetUpVertex[MinSize=2pt] \grTutteCoxeter[RA=5]%,RB=1]

\draw (a0) node[above]{$w_{2,5}$};
\draw (a1) node[above]{$l$};
\draw (a2) node[above]{$d$};
\draw (a3) node[above]{$p$};
\draw (a4) node[above left]{$e$};
\draw (a5) node[above left]{$i$};
\draw (a6) node[above left]{$a$};
\draw (a7) node[left]{$w_{2,0}$};
\draw (a8) node[below left]{$w_{1,0}$};
\draw (a9) node[below left]{$w_{2,1}$};
\draw (a10) node[below left]{$c$};
\draw (a11) node[below left]{$n$};
\draw (a12) node[left]{$f$};
\draw (a13) node[below left]{$w_{2,2}$};

\draw (a14) node[below]{$w_{1,1}$};
\draw (a15) node[below right]{$w_{0,0}$};
\draw (a16) node[below right]{$w_{0,1}$};
\draw (a17) node[below right]{$w_{1,2}$};
\draw (a18) node[below right]{$w_{2,4}$};
\draw (a19) node[right]{$j$};
\draw (a20) node[right]{$h$};
\draw (a21) node[right]{$w_{2,3}$};
\draw (a22) node[right]{$g$};
\draw (a23) node[right]{$m$};
\draw (a24) node[right]{$w_{2,6}$};
\draw (a25) node[right]{$w_{1,3}$};
\draw (a26) node[right]{$w_{2,7}$};
\draw (a27) node[above right]{$o$};
\draw (a28) node[above right]{$b$};
\draw (a29) node[above right]{$k$};

  \end{tikzpicture}
 \end{minipage}
\end{figure}

\end{frame}



\begin{frame}{Jaulas c�bica}



\only<2>{
\begin{itemize}
\item
{Biggs y Hoare}
\item
{ Brendan McCay}
\item
{Brinkmann y Saager}
\end{itemize}
}


\only<3>{
\begin{itemize}
\item
{M. O'Keefe y P.K. Wong}
\end{itemize}
}


\only<4>{
\begin{itemize}
\item
{Brendan McKay y Wendy Myrvold}

\end{itemize}
}


\only<6>{
\begin{itemize}
\item
{Brendan McKay y Wendy Myrvold}
\item{Miles Whitehead}
\end{itemize}
}


\begin{table}
\centering
\begin{tabular}{cccc}
\text{Jaula} & Cantidad & Cota de Moore & La m�s chica\\
\hline
\only<1-6>{
$(3,3)$-jaula &1 & 4&4\\
$(3,4)$-jaula & 1 &6&6\\
$(3,5)$-jaula & 1 &10&10\\}
$(3,6)$-jaula& 1  &14&14\\
$(3,7)$-jaula& 1  &22&24\\
$(3,8)$-jaula& 1  &30&30\\
\pause 
\alert<2>{$(3,9)$-jaula}&18 & 46&58\\
\pause
\alert<3>{$(3,10)$-jaula}& 3 &62 &70\\
\pause
\alert<4>{$(3,11)$-jaula}& 1 & 94 &112\\
\pause
\alert<5>{$(3,12)$-jaula}& 1 & 126&126\\
\pause
\alert<6>{$(3,13)$-jaula}&1+&190[202]&272\\
\pause
$(3,14)$-jaula & 1+ & 254[258] & 384\\
$(3,15)$-jaula & 1+ & 382 & 620\\
$(3,16)$-jaula & 1+ & 510 & 960\\
$(3,17)$-jaula & 1+ & 766 & 2176\\
$(3,18)$-jaula & 1+ & 1022 & 2640\\
$(3,19)$-jaula & 1+ & 1534 & 4324\\
$(3,20)$-jaula & 1+ & 2046 & 6048\\
$(3,21)$-jaula & 1+ & 3070 & 16028\\
$(3,22)$-jaula & 1+ & 4094 & 16206\\
%$(3,23)$-jaula & 1+ & 6142 & 49482\\

\end{tabular}
\end{table}

\end{frame}

\begin{frame}{Un m�todo}
\only<1>{
\begin{block}{}
Idear una forma de construir gr�ficas c�bicas
\end{block}}
\begin{itemize}
\item Las jaulas de cuello par, que hasta el momento se conocen son bipartitas.
\item La gr�fica bipartita clanica del complemento de la gr�fica de lineas de $K_6$ es la $(3,8)$-jaula.
\end{itemize}

\only<2-5>{
\begin{columns}
\column{.5\textwidth}
\only<2-5>{
\begin{figure}
\centering
\begin{tikzpicture}[scale=0.4,rotate=90]
    \SetUpVertex[MinSize=10pt]

    \SetVertexNoLabel{\grComplete[prefix=v,RA=3]{6}}
\foreach \y in {0,1,...,5} { \setcounter{in}{\y+1} \stepcounter{in}
  \draw (v\y) node{\y}; }
\end{tikzpicture}
\end{figure}
}
\column{.6\textwidth}
\only<3-5>{
\begin{figure}
  \centering
  \begin{tikzpicture}[scale=0.5,rotate=90]
    \SetUpVertex[MinSize=.1pt]
    \SetVertexNoLabel{\grEmptyCycle[prefix=v,RA=3]{15}}

 % { \foreach
 %      \y in {0,1,...,5} { \setcounter{in}{\y+1} \stepcounter{in} \draw
 %        (v\y) node[above]{{in}}; }}

% \Vertex[x=-0,y=0]{w0} }
  %  \Edges(w0,v1,v2,w0,v3,v4,w0,v5,v0,w0)

    \draw (v0) node[above]{$(12)$};

\only<4-5>{\AddVertexColor{black}{v0}}

\only<3,4>{
\draw (v1) node[left]{$(13)$};
\draw (v2) node[ left]{$(14)$};
\draw (v3) node[left]{$(15)$};
\draw (v4) node[left]{$(16)$};
\draw (v11) node[right]{$(13)$};
\draw (v12) node[ right]{$(14)$};
\draw (v13) node[right]{$(15)$};
\draw (v14) node[right]{$(16)$};
}
\only<3,5>{
\draw (v5) node[left]{$(56)$};
\draw (v6) node[left]{$(34)$};
\draw (v7) node[below left]{$(35)$};
\draw (v8) node[below right]{$(46)$};
\draw (v9) node[right]{$(45)$};
\draw (v10) node[right]{$(36)$};
}

\only<4>{
\Edges[color=BlueViolet](v1,v0,v2)
\Edges[color=BlueViolet](v3,v0,v4)
\Edges[color=BlueViolet](v11,v0,v12)
\Edges[color=BlueViolet](v13,v0,v14)
\AddVertexColor{blue}{v1,v2,v3,v4,v11,v12,v13,v14}}


\only<5>{
\AddVertexColor{violet}{v5,v6,v7,v8,v9,v10}
\Edges[color=Violet](v5,v0,v6)
\Edges[color=Violet](v7,v0,v8)
\Edges[color=Violet](v9,v0,v10)
}
%\AddVertexColor{}{}


  \end{tikzpicture}
\end{figure}
}
\end{columns}
}

\only<6-7>{
\begin{figure}
  \centering
  \begin{tikzpicture}[scale=.7]
    \SetVertexNoLabel 
    \grEmptyCycle[prefix=v,RA=3]{6}
      \Vertex[x=0,y=0]{w0} 
   
 %\Edges(w0,v1,v2,w0,v0,v5,w0,v3,v4,w0)
\Edges[color=Violet,style={line width=1.5pt}](v0,w0,v5)
\Edges[color=Violet,style={line width=1.5pt}](v1,w0,v2)
\Edges[color=Violet,style={line width=1.5pt}](v3,w0,v4)

\only<7>{
\Edges[color=Violet,style={line width=1.5pt}](v0,v5)
\Edges[color=Violet,style={line width=1.5pt}](v1,v2)
\Edges[color=Violet,style={line width=1.5pt}](v3,v4)

}
   \draw (w0) node{$12$};
 \draw (v2) node{$34$}; \draw (v1) node{$56$};
    \draw (v0) node{$36$}; \draw (v5) node {$45$}; \draw (v3)  node{$35$}; \draw (v4) node {$46$};

  \end{tikzpicture}
%\only<7>{
%  \caption{Vecindad cerrada de $\{1,2\}$.}\label{vecindadde12}}
\end{figure}
}
\end{frame}

\end{document}




\begin{frame}
  \frametitle{Flores}

\begin{columns}
\begin{column}{5cm}\vspace*{1cm}

  \begin{itemize}
  \item<1-|alert@1>{Rosa}
  \item<2->{Azucena}
  \item<3-|alert@3>{Margarita}
  \end{itemize}


\end{column}


\begin{column}{5cm}
\begin{center}

\only<1>{hola aqui estoy}
\only<3>{
\begin{tabular}{|c|c|c|}
\hline &H.S.$M$&No H.S.$M$
\\ \hline
H.S.$M^{\perp}$&$Ex3$&$Ex4$
\\ \hline
No H.S.$M^{\perp}$&$Ex4$&$\smiley$
\\ \hline
\end{tabular}
}

\end{center}
\end{column}
\end{columns}
\end{frame}


\end{document}


\begin{frame}

\begin{alertblock}<+->{Definici�n}
Un operador  lineal acotado $T:\mathcal{H}\to\mathcal{H}$ se dice
\textbf{hiperciclico} si para alg�n $x\in H$, $\text{Orb}(T,x)$ es
densa en $\mathcal{H}$.
\end{alertblock}

\begin{alertblock}<+->{Definici�n}
Un operador  lineal acotado $T:\mathcal{H}\to\mathcal{H}$ es
\textbf{hiperciclico en un subespacio M} si existe un vector
$x\in\mathcal{H}$ tal que $\text{Orb}(T,x)\bigcap M$ sea densa en
$M$.
\end{alertblock}

\end{frame}

\begin{frame}

\begin{prop}<+->[Kitai, 1982]
Si $T$ es invertible, $T$ es hiperciclico si y solo si $T^{-1}$ lo
es.
\end{prop}
\end{frame}

\begin{frame}


\begin{alertblock}<+->{Definici�n}
Un \textbf{desplazamiento bilateral con pesos } es un operador $T$
donde $$T(\ldots,x_{-1},x_0,x_1,\ldots ) =
(\ldots,a_{-2}x_{-1},a_{-1}x_0,a_0x_1,a_1x_2,a_2x_3,\ldots )$$ con
$\{a_n\}_{n\in \Z}$ una sucesi�n acotada de n�meros complejos.
\end{alertblock}


\end{frame}




\begin{frame}
{ Para un subespacio $M$ de dimensi�n y codimenci�n  infinitas que
tipo de operadores invertibles existen?}

\textbf{Hiperciclico en $\mathcal{H}$}
\begin{tabular}{|c|c|c|}
\hline &H.S.$M$ & No H.S.$M$
\\ \hline
H.S. $M^{\perp}$&$\smiley$&$?$
\\ \hline
No H.S.$M^{\perp}$&$?$&$?$
\\ \hline
\end{tabular}

\vspace{15pt}

\textbf{No Hiperciclico en $\mathcal{H}$}
\begin{tabular}{|c|c|c|}
\hline &H.S.$M$&No H.S.$M$
\\ \hline
H.S.$M^{\perp}$&$?$&$?$
\\ \hline
No H.S.$M^{\perp}$&$?$&$\smiley$
\\ \hline
\end{tabular}
\end{frame}



\begin{frame}
\begin{example}
Define $A=\{\pm 0, \pm 1\}\bigcup\{\pm 4,\pm 6,\pm 8,\ldots\}$. Sea
$M=\text{span}\{e_i\}_{|i|\in A}$.Entonces
$M^{\perp}=\text{span}\{e_i\}_{|i|\notin A}$ Los elementos de $M$
son de la forma $$x=(\ldots
,x_{-3},0,x_{-2},0,0,x_{-1},x_0,x_1,0,0,x_2,0,x_3\ldots )$$ y los
elementos de $M^{\perp}$ son $$y=(\ldots
,0,y_{-3},0,y_{-2},y_{-1},0,0,0,y_1,y_2,0,y_3,0\ldots )$$ Entonces
ning�n desplazamiento bilateral con pesos hiperciclico es
hipersiciclo en un subespacio $M$ o $M^{\perp}$.
\end{example}

\textbf{Hiperciclico en $\mathcal{H}$}
\begin{tabular}{|c|c|c|}
\hline &H.S.$M$ & No H.S.$M$
\\ \hline
H.S. $M^{\perp}$&$\smiley$&$?$
\\ \hline
No H.S.$M^{\perp}$&$?$&$Ex1$
\\ \hline
\end{tabular}


\end{frame}


\begin{frame}
\begin{example}
Define $A=\{0\}\bigcup\{\pm1,\pm3,\pm5,\ldots\}$. Sea
$M=\text{span}\{e_i\}_{|i|\in A}$. Entonces
$M^{\perp}=\text{span}\{e_i\}_{|i|\notin A}$. Los elementos de $M$
son de la forma
$$x=(\ldots ,0,x_{-2},0,x_{-1},0,0,0,x_1,0,x_2,0,\ldots )$$ y los de $M^{\perp}$ son $$y=(\ldots
,y_{-3},0,y_{-2},0,y_{-1},\mathbf{y_0},y_1,0,y_2,0,y_3,\ldots )$$
Entonces para alg�n $T$ un desplazamiento bilateral con pesos
hiperciclico, $T$ es hiperciclico en  $M$ pero no hiperciclico en
$M^{\perp}$.
\end{example}

\textbf{Hiperciclico en $\mathcal{H}$}
\begin{tabular}{|c|c|c|}
\hline &H.S.$M$ & No H.S.$M$
\\ \hline
H.S. $M^{\perp}$&$\smiley$&$Ex2$
\\ \hline
No H.S.$M^{\perp}$&$Ex2$&$Ex1$
\\ \hline
\end{tabular}

\end{frame}


\begin{frame}

\begin{thm}[Salas, 1995]
Sea $T$ un desplazamiento bilateral con pesos hacia atr�s en
$\ell^2(\Z)$ con su sucesi�n de pesos positivos y acotados
$\{a_n\}$. Entonces $T$ es hiperciclico si y solo si dado $\epsilon
>0 $y $q\in\Z $, existe una n arbitrariamente grande tal que para todo $\mid j\mid \leq
q$,


 $\Pi_{s=0}^{n-1}a_{s+j}>1/\epsilon $ y $\Pi_{s=1}^{n}a_{s-j}<\epsilon $

\end{thm}

\begin{thm}[Salas, 1995]

Sean $T_1$ y $T_2$ desplazamientos bilaterales con pesos con
sucesi�n de pesos $\{a_n\}$ y $\{b_n\}$. $T_1 \oplus T_2$ es
hiperciclico si y solo si  dado $\epsilon > 0$ y $q \in \N$ existe
una n suficientemente grande tal que para todo $\mid j\mid \leq q$

$$
\max \left\{ \prod_{k=0}^{n-1} a_{j+k}, \prod_{k=0}^{n-1} b_{j+k}
\right\} < \epsilon ,\hspace{10pt} \min \left\{ \prod_{k=1}^{n}
a_{j-k}, \prod_{k=1}^{n} b_{j-k} \right\} > 1/\epsilon
$$

\end{thm}

\end{frame}




\begin{frame}
\begin{example}
Define dos sucesiones bilaterales $$\{a_n\}_{n\in\Z}=\left\{\ldots
,\frac{1}{2},\frac{1}{2},\frac{1}{2},2,2,\frac{1}{2},\textbf{1},2,\frac{1}{2},\frac{1}{2},2,2,2,\ldots\right\}$$
$$\{b_n\}_{n\in\Z}=\left\{\ldots
,2,2,2,\frac{1}{2},\frac{1}{2},2,\textbf{1},\frac{1}{2},2,2,\frac{1}{2},\frac{1}{2},\frac{1}{2},\ldots\right\}$$
Define un operador $T=P\oplus R:\mathcal{H}_1\oplus\mathcal{H}_2\to
\mathcal{H}_1\oplus\mathcal{H}_2$, donde $(Px)_i = a_i x_{i+1}$ y
$(Rx)_i = b_i x_{i+1}$. $T$ es hiperciclico en el subespacio
$\mathcal{H}_1$ y en $\mathcal{H}_2$, pero no es hiperciclico en
todo el espacio $\mathcal{H}_1\oplus\mathcal{H}_2$.

\end{example}


\textbf{No Hiperciclico en $\mathcal{H}$}
\begin{tabular}{|c|c|c|}
\hline &H.S.$M$&No H.S.$M$
\\ \hline
H.S.$M^{\perp}$&$Ex3$&$?$
\\ \hline
No H.S.$M^{\perp}$&$?$&$\smiley$
\\ \hline
\end{tabular}

\end{frame}


\begin{frame}
\begin{example}
Sea $M=\text{span}\{e_i\}_{i\text{impar}}$. Si $x\in M$, $x$ es de
la forma $x=(\ldots ,0,x_{-2},0,x_{-1},0,x_1,0,x_2,0,\ldots)$.
Entonces $M^{\perp}=\text{span}\{e_i\}_{i\text{par}}$. Define
$T:\ell^2\to\ell^2$ tal que
\[(Tx)_i=
\left\{
\begin{array}{lc}
x_i&i \text{par}\\
a_i x_{i+2}&i \text{impar}
\end{array}
\right. \text{donde\ } a_i=\left\{
\begin{array}{ll}
3&i>0\\
\frac{1}{2}&i<0
\end{array}
\right.
\]
Esto es, si $x=(\ldots ,x_{-3},x_{-2},x_{-1},x_0,x_1,x_2,x_3,\ldots)$ entonces $Tx=(\ldots ,\frac{1}{2}x_{-1},x_{-2},\frac{1}{2}x_{1},x_0,3x_3,x_2,3x_5,\ldots)$.\\
Entonces, $T$ es hiperciclico en el subespacio $M$ pero no
hiperciclico y no hiperciclico en el subespacio $M^{\perp}$.

\end{example}

\textbf{No Hiperciclico en $\mathcal{H}$}
\begin{tabular}{|c|c|c|}
\hline &H.S.$M$&No H.S.$M$
\\ \hline
H.S.$M^{\perp}$&$Ex3$&$Ex4$
\\ \hline
No H.S.$M^{\perp}$&$Ex4$&$\smiley$
\\ \hline
\end{tabular}

\end{frame}


\section{Mas preguntas}

\begin{frame}

\setcounter{questioncount}{0}

\begin{question}
Si un operador invertible $T$ es hiperciclico en un subespacio $M$,
es $T^{-1}$ hiperciclico en el subespacio $M$?
\end{question}

\pause

\begin{question}
Si $T$ es hiperciclico en un subespacio $M$, es $T^n$ hiperciclico
en el subespacio $M$?
\end{question}

\pause


\begin{question}
Cada operador hiperciclico tiene un subespacio de codimensi�n
infinita en donde este sea hiperciclico en el subespacio?
\end{question}

\end{frame}


\begin{frame}
\setbeamercolor{block title}{bg=cyan}
\begin{block}{Referencias}
\begin{enumerate}
\footnotesize
\item S. Grivaux, \textit{Construction of operators with prescribed behaviour,} Arch. Math. \textbf{81} (2003), 291-299.
\item C. Kitai, \textit{Invariant closed sets for linear operators,} Thesis, Univ. of Toronto, Toronto, 1982.
\item C. Le, \textit{On subspace hypercyclic operators,} private communication.
\item B. F. Madore and R. A. Martinez-Avendano, \textit{Subspace Hypercyclicity,} 2010.
\item S. Rolewicz, \textit{On orbits of elements,} Studia Math. \textbf{32} (1969), 17-22.
\item H. Salas, \textit{Hypercyclic weighted shifts,} Trans. Amer. Math. Soc. \textbf{347} (1995), 993-1004.
\item K. Saxe (2010) \textit{Beginning Functional Analysis,} Springer-Verlag New York, Inc.
\item J. Shapiro (2001), \textit{Notes on the dynamics of linear operators,} Unpublished lecture notes (available online at www.math.msu.edu/~shapiro).
\item N. Young (1988), \textit{An Introduction to Hilbert Space,} Cambridge University Press.
\end{enumerate}
\end{block}
\end{frame}


\section{La experiencia}

\begin{frame}
{Forma de trabajo}
\begin{enumerate}
\item 4 grupos de trabajo de entre 3 y 4 personas mas un asesor
\pause
\item Horario de 10am a 5pm con una hora para almorzar
\pause
\item Una platica a fines del primer mes y una a fines del segundo
para reportar avances
\pause
\item Platicas cada semana con profesores invitados
\end{enumerate}
\begin{figure}[ht]
\begin{center}
\includegraphics[height=4.4057cm,width=5.8205cm]{grupo.jpg}
\end{center}
\end{figure}
\end{frame}



\begin{frame}
{Cosas notables del lugar y sus habitantes}
\begin{enumerate}
\item Comida
\pause
\item Amabilidad
\pause
\item Salidas a caminatas muy largas
\end{enumerate}
\begin{figure}[ht]
\begin{center}
\includegraphics[height=4.4057cm,width=5.8205cm]{caminar.jpg}
\includegraphics[height=4.4057cm,width=5.8205cm]{camina.jpg}
\end{center}
\end{figure}

\end{frame}

\begin{frame}
{Paseos}

\begin{figure}[ht]
\begin{center}
\includegraphics[height=3.5cm,width=5cm]{cbo.jpg}
\includegraphics[height=3.5cm,width=5cm]{cr.jpg}
\end{center}
\end{figure}

\begin{figure}[ht]
\begin{center}
\includegraphics[height=4cm,width=5cm]{rb.jpg}
\includegraphics[height=4cm,width=4cm]{ota.jpg}
\end{center}
\end{figure}

\end{frame}


\begin{frame}
{Gracias}
\begin{figure}[ht]
\begin{center}
\includegraphics[height=7.5cm,width=10cm]{grac.jpg}
\end{center}
\end{figure}


\end{frame}

\end{document}
